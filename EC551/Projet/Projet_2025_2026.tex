\documentclass[a4paper,12pt]{article}
\usepackage{amssymb,amsmath,amsthm,dsfont,mathtools,stmaryrd}

%-- Just labeled equations
\usepackage{mathtools}
\mathtoolsset{showonlyrefs}

%-- Language parameters
\usepackage[utf8]{inputenc}
\usepackage[french]{babel}

%-- Font parameters
\usepackage{lmodern}
\usepackage{csquotes} % "guillemets"
\usepackage{color} % For colors

%-- Margin parameters
\usepackage{geometry}
\geometry{vmargin=1.5cm,hmargin=1cm}

%-- For importing graphics
\usepackage{graphicx}
\graphicspath{{Images/}}

%-- For image alignment
\usepackage[export]{adjustbox}

%-- Better float management
\usepackage{float}

%-- For subfigures
\usepackage{subcaption}

%-- Paragraph parameters 
\setlength{\parindent}{0pt}
\setlength{\parskip}{3pt}

%-- Enumeration / List parameters
\usepackage{enumitem}


%-- URL parameters
\usepackage{hyperref}
%\usepackage{cleveref}
\hypersetup{colorlinks,
breaklinks,
urlcolor=blue,
linkcolor=black,
citecolor=blue}

%-- For arrays

\usepackage{arydshln}

%-- Math commands / symbols !!!! After URL parameters


% Common sets
\newcommand{\N}{\mathbb{N}}
\newcommand{\Z}{\mathbb{Z}}
\newcommand{\R}{\mathbb{R}}
\newcommand{\X}{\mathcal{X}}

% Integration
\renewcommand{\d}{\mathrm{d}}

% Exponential
\newcommand{\e}{\mathrm{e}}


% Index function
\newcommand{\ind}{\mathds{1}}


% Inner product
\newcommand{\scal}[2]{\langle #1 , #2 \rangle}
% Diverse
\newcommand{\argmin}{\mathrm{argmin}}
\newcommand{\argmax}{\mathrm{argmax}}
\renewcommand{\P}{\mathbb{P}}
\newcommand{\E}{\mathbb{E}}



%-- Environments
\newcounter{numexo}
\setcounter{numexo}{0}
\newenvironment{exo}[1][]{\stepcounter{numexo} \textbf{Exercice \thenumexo} #1}{\medskip}
\usepackage{multicol}

%-- Titre

\title{\begin{normalsize}\textbf{
\includegraphics[scale=0.25]{logo_oniris.png} \\
EC 551 : Statistique descriptive et décisionnelle
 \\ Projet}
\end{normalsize}}
\author{{\small Guillaume Franchi}}
\date{{\small Année universitaire 2025-2026}}

%-- Sous-titres
\usepackage[svgnames]{xcolor}
\newcommand{\greysquare}{\textcolor{gray}{$\blacksquare$}}

\makeatletter
  \def\vhrulefill#1{\leavevmode\leaders\hrule\@height#1\hfill \kern\z@}
\makeatother

\newcommand{\sstitre}[1]{\greysquare \ \textbf{\textsf{{\Large #1}}} \textcolor{gray}{\vhrulefill{2pt}} \medskip }

%-- changement puces
\AtBeginDocument{\renewcommand{\labelenumi}{\textbf{\arabic*})}}

\AtBeginDocument{\renewcommand{\labelenumii}{\textbf{\alph*})}}

%-- POur tableau double entrée
\usepackage{nicematrix}

\begin{document}

\maketitle

\sstitre{Présentation du projet}

Le jeu de données \texttt{olive$\_$oil.csv}, disponible sur connect, contient des données relatives aux compositions en acides gras et au prix au litre de 572 huiles d'olives provenant de différentes régions d'Italie.

\medskip

\emph{Remarque : les données ont été simulées à des fins pédagogiques, et ne reflètent pas la réalité du marché.}

\medskip

On s'intéresse aux variables pouvant expliquer le montant payé par un assuré.

\medskip

Par binôme, vous devrez rendre sur connect pour le \textbf{mercredi 22 octobre 12h00}:
	\begin{enumerate}[label=$\bullet$]
	\item un compte-rendu \emph{(format : CR$\_$nombinome1$\_$nombinome2.pdf)};
	\item un script \textsf{R} \emph{(format : script$\_$nombinome1$\_$nombinome2.R)};
	\end{enumerate}

permettant de répondre aux questions posées ci-dessous. On s'appuiera sur des représentations graphiques et des indicateurs statistiques présentés en cours.

\medskip

\emph{Remarque : Avant d'analyser les possibles liens entre deux variables, on effectuera une rapide analyse univariée des variables considérées.}

\medskip

\sstitre{Questions}

	\begin{enumerate}
	\item Quelles compositions en acides gras semblent expliquer le prix de l'huile d'olive ?
	\item La région de provenance de l'huile semble-t-elle expliquer sa composition en acide linoléique ?
	\item Créer une nouvelle variable qualitative correspondant à une gamme de prix :
	\begin{enumerate}[label=$\bullet$]
	\item $<6$ : entrée de gamme;
	\item $[6,8[$: bon marché;
	\item $[8,10[$: moyenne gamme;
	\item $[10,12[$: gamme supérieure 
	\item $\geqslant 12$: haut de gamme.
	\end{enumerate}
	Existe-t-il un lien entre ces différentes gammes et la région de provenance de l'huile d'olive ?
	\end{enumerate}

\end{document}